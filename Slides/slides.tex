%%%%%%%%%%%%%%%%%%%%%%%%%%%%%%%%%%%%%%%%%
% Beamer Presentation
% LaTeX Template
% Version 1.0 (10/11/12)
%
% This template has been downloaded from:
% http://www.LaTeXTemplates.com
%
% License:
% CC BY-NC-SA 3.0 (http://creativecommons.org/licenses/by-nc-sa/3.0/)
%
%%%%%%%%%%%%%%%%%%%%%%%%%%%%%%%%%%%%%%%%%

%----------------------------------------------------------------------------------------
%	PACKAGES AND THEMES
%----------------------------------------------------------------------------------------

\documentclass{beamer}

\mode<presentation> {

% The Beamer class comes with a number of default slide themes
% which change the colors and layouts of slides. Below this is a list
% of all the themes, uncomment each in turn to see what they look like.

%\usetheme{default}
%\usetheme{AnnArbor}
%\usetheme{Antibes}
%\usetheme{Bergen}
%\usetheme{Berkeley}
%\usetheme{Berlin}
%\usetheme{Boadilla}
%\usetheme{CambridgeUS}
%\usetheme{Copenhagen}
%\usetheme{Darmstadt}
%\usetheme{Dresden}
%\usetheme{Frankfurt}
%\usetheme{Goettingen}
%\usetheme{Hannover}
%\usetheme{Ilmenau}
%\usetheme{JuanLesPins}
%\usetheme{Luebeck}
%\usetheme{Madrid}
%\usetheme{Malmoe}
%\usetheme{Marburg}
%\usetheme{Montpellier}
%\usetheme{PaloAlto}
%\usetheme{Pittsburgh}
%\usetheme{Rochester}
%\usetheme{Singapore}
%\usetheme{Szeged}
\usetheme{Warsaw}

% As well as themes, the Beamer class has a number of color themes
% for any slide theme. Uncomment each of these in turn to see how it
% changes the colors of your current slide theme.

%\usecolortheme{albatross}
%\usecolortheme{beaver}
%\usecolortheme{beetle}
\usecolortheme{crane}
%\usecolortheme{dolphin}
%\usecolortheme{dove}
%\usecolortheme{fly}
%\usecolortheme{lily}
%\usecolortheme{orchid}
%\usecolortheme{rose}
%\usecolortheme{seagull}
%\usecolortheme{seahorse}
%\usecolortheme{whale}
%\usecolortheme{wolverine}

%\setbeamertemplate{footline} % To remove the footer line in all slides uncomment this line
%\setbeamertemplate{footline}[page number] % To replace the footer line in all slides with a simple slide count uncomment this line

%\setbeamertemplate{navigation symbols}{} % To remove the navigation symbols from the bottom of all slides uncomment this line
}

\usepackage{graphicx} % Allows including images
\usepackage{booktabs} % Allows the use of \toprule, \midrule and \bottomrule in tables
%\usepackage[utf8]{inputenc} %Codificacion utf-8
%\newtheorem{example}{ejemplo}
\usepackage[T1]{fontenc}
\usepackage[utf8]{inputenc}
\usepackage[spanish]{babel} %Definir idioma español
\uselanguage{Spanish}
\languagepath{Spanish}

%\usetheme{Copenhagen}

\deftranslation[to=Spanish]{Example}{Ejemplo}

%comands
\usepackage{algorithm}
\usepackage[noend]{algpseudocode}
\def\BState{\State\hskip-\ALG@thistlm}
\def\BSState{\State\hskip\hskip-\ALG@thistlm}
\newcommand*{\escape}[1]{\texttt{\textbackslash#1}}

%----------------------------------------------------------------------------------------
%	TITLE PAGE
%----------------------------------------------------------------------------------------

\title[Tokenizador, Pos tagger y lematizador]{Algoritmos para el procesamiento del lenguaje natural en Español: Tokenizador, POS tagger y lematizador} % The short title appears at the bottom of every slide, the full title is only on the title page

\author{Cristina Heredia} % Your name
\institute[UGR] % Your institution as it will appear on the bottom of every slide, may be shorthand to save space
{
Trabajo fin de grado en Ingeniería informática \\ % Your institution for the title page
\medskip
\textit{Universidad de Granada} % Your email address
}
\date{\today} % Date, can be changed to a custom date

\begin{document}

\begin{frame}
\titlepage % Print the title page as the first slide
\end{frame}

%\begin{frame}
%\frametitle{Overview} % Table of contents slide, comment this block out to remove it
%\tableofcontents % Throughout your presentation, if you choose to use \section{} and \subsection{} commands, these will automatically be printed on this slide as an overview of your presentation
%\end{frame}

%----------------------------------------------------------------------------------------
%	PRESENTATION SLIDES
%----------------------------------------------------------------------------------------

%------------------------------------------------
\section{Introdución al problema } % Sections can be created in order to organize your presentation into discrete blocks, all sections and subsections are automatically printed in the table of contents as an overview of the talk
%------------------------------------------------




%\-----------------------------
%\  Introducción al PLN
%\-----------------------------
\begin{frame}
\frametitle{Introducción al PLN}
\begin{block}{Definición}
PLN es el área de estudio y aplicación que engloba cualquier manipulación computacional del lenguaje natural. 
\end{block}
%\begin{frame}[fragile] % Need to use the fragile option when verbatim is used in the slide
%\frametitle{Verbatim}
\begin{example}[Aplicaciones de PLN]
Contar ocurrencias de las palabras para detectar estilo de escritura, asistentes de voz (Siri, Google Now)...
\end{example}

\begin{block}{Áreas de aplicación del PLN}
\begin{itemize}
\item Traducción automática
\item Sistemas de reconocimiento del habla
\item Lectura automática
\item Minería de datos en RRSS
\item Análisis de sentimientos
\end{itemize}
\end{block}
\end{frame}

%------------------------------------------------
%\  Análisis de sentimientos- estado del arte
%\-----------------------------------------------
\begin{frame}
\frametitle{Análisis de sentimientos-Estado del arte}
\begin{itemize}
\item A nivel de documento
\item A nivel de oración
\item A nivel de aspecto
%\item Nam cursus est eget velit posuere pellentesque
%\item Vestibulum faucibus velit a augue condimentum quis convallis nulla gravida
\end{itemize}
\end{frame}
\subsection{El pipeline} % A subsection can be created just before a set of slides with a common theme to further break down your presentation into chunks
%------------------------------------------------
%\  Introducción al problema (motivación)
%\-----------------------------------------------
\begin{frame}
\frametitle{Introducción al problema}
El objetivo final de toda suite PLN es tener un conjunto de herramientas para realizar finalmente análisis de sentimientos o alguna extracción de información.
\\~\\
\textbf{Procesado del texto}
\begin{enumerate}
\item Tokenización
\item Detección de límites de oraciones
\item Lematización
\item Stemming
\item POS tagging
\item Parseo sintáctico
\end{enumerate}

%\end{columns}
%\begin{block}{Block 1}
%Lorem ipsum dolor sit amet, consectetur adipiscing elit. Integer lectus nisl, ultricies in feugiat rutrum, p%orttitor sit amet augue. Aliquam ut tortor mauris. Sed volutpat ante purus, quis accumsan dolor.
%\end{block}

%\begin{block}{Block 2}
%Pellentesque sed tellus purus. Class aptent taciti sociosqu ad litora torquent per conubia nostra, per inceptos %himenaeos. Vestibulum quis magna at risus dictum tempor eu vitae velit.
%\end{block}

%\begin{block}{Block 3}
%Suspendisse tincidunt sagittis gravida. Curabitur condimentum, enim sed venenatis rutrum, ipsum neque consectetur %orci, sed blandit justo nisi ac lacus.
%\end{block}
\end{frame}

%------------------------------------------------
%\  tokenización
%\-----------------------------------------------
\begin{frame}
\frametitle{Introducción al problema: Tokenización}
\begin{block}{Definición}
Proceso que separa el texto en uniades atómicas, los tokens(palabras, números, símbolos).
\end{block}

\begin{block}{Técnicas}
Freeling, Apache OpenNLP ,NLTK, StandfordNLP
\end{block}

%\begin{block}{Block 3}
%Suspendisse tincidunt sagittis gravida. Curabitur condimentum, enim sed venenatis rutrum, ipsum neque consectetur %orci, sed blandit justo nisi ac lacus.
%\end{block}
\end{frame}

%------------------------------------------------
%\  figura coreNLP
%\-----------------------------------------------
\begin{frame}
\frametitle{Ejemplo de pipeline: Stanford CoreNLP}
\begin{figure}
\includegraphics[width=0.45\linewidth]{../Memory/Dissertate-Harvard-LaTeX/stanford.png}
\end{figure}

%\begin{block}{Block 3}
%Suspendisse tincidunt sagittis gravida. Curabitur condimentum, enim sed venenatis rutrum, ipsum neque consectetur %orci, sed blandit justo nisi ac lacus.
%\end{block}
\end{frame}

%------------------------------------------------
\section{Resolución del trabajo}
%------------------------------------------------
\subsection{Metodología}
%-------------Scala--------------------------
\begin{frame}
\frametitle{Un lenguaje de programación: Scala}
\begin{columns}[c] % The "c" option specifies centered vertical alignment while the "t" option is used for top vertical alignment

\column{.45\textwidth} % Left column and width
\textbf{Razones para aprender Scala}
\begin{enumerate}
\item Es Scalable
\item Paradigma mixto
\item Sistema amplio de tipos
\item Herramientas y librería de Java
\item Estáticamente tipado
\item Sintaxis concisa, elegante, flexible
\item Scala en concurrencia y big data
\end{enumerate}

\column{.5\textwidth} % Right column and width
Lorem ipsum dolor sit amet, consectetur adipiscing elit. Integer lectus nisl, ultricies in feugiat rutrum, porttitor sit amet augue. Aliquam ut tortor mauris. Sed volutpat ante purus, quis accumsan dolor.

\end{columns}
\end{frame}
%------------------transparencia TDD---------
\begin{frame}
\frametitle{Test driven development - TDD }
\begin{columns}[c] % The "c" option specifies centered vertical alignment while the "t" option is used for top vertical alignment

\column{.45\textwidth} % Left column and width
\textbf{Mencionar Scalatest y Funsite y TDD}
%\begin{enumerate}
%\item Es Scalable
%\item Paradigma mixto
%\item Sistema amplio de tipos
%\item Herramientas y librería de Java
%\item Estáticamente tipado
%\item Sintaxis concisa, elegante, flexible
%\end{enumerate}

\column{.5\textwidth} % Right column and width
Lorem ipsum dolor sit amet, consectetur adipiscing elit. Integer lectus nisl, ultricies in feugiat rutrum, porttitor sit amet augue. Aliquam ut tortor mauris. Sed volutpat ante purus, quis accumsan dolor.

\end{columns}
\end{frame}
%-------------Diagramas------------------- 
\begin{frame}
\frametitle{¿Diagrama de las clases?}
\end{frame}
%---------------material---------------------
\subsection{Material empleado}
\begin{frame}
\frametitle{Material empleado: POS tagger}
\textbf{¿Pongo lectura de Wikicorpus o solo lo menciono?}
\end{frame}

%---------------obtención del léxico-----------
\begin{frame}[fragile]
\frametitle{POS tagger - Obtención del léxico}
\begin{columns}[t] % The "c" option specifies centered vertical alignment while the "t" option is used for top vertical alignment
\column{.40\textwidth} % Left column and width
\textbf{Pasos:}
\begin{enumerate}
\item Contar ocurrencia de cada palabra en corpus
\item Contar ocurrencia de las etiquetas. Hay palabras que
pueden tener varias etiquetas
\item Construir diccionario tomando las palabras más frecuentes y su etiqueta más frecuente
\end{enumerate}
\column{.55\textwidth} % Left column and width
\begin{algorithm}[H]
    \begin{algorithmic}[1]
    \algrenewcommand\alglinenumber[1]{\tiny #1:}
    \Procedure{createLexicon(corpus, maxLines=100000)}{}
    \State $\text{Lexicon} \gets$ \o
    \State $\text{top} \gets$ \o
    \For{$\text{sentence in corpus}$}
            \For{$\text{w, tag in sentence}$}
            \State $\text{increase ocurrences of Lexicon(w, tag)} $
            \EndFor
    \EndFor        
    \For{$\text{w, tags in Lexicon}$} 
            \State $\text{freq} \gets \text{$\sum$ all ocurrences of tags for w}$
             \State $\text{tag} \gets \text{the tag with bigger ocurrence for w}$
             \State $\text{top} \gets \text{top + (freq,w,tag) }$
            \EndFor
            
     \State $\text{get the maxLines with bigger ocurrency}$ 
     \State $\text{write to file}$      
        \EndProcedure
    \end{algorithmic}
    \label{alg:rAP}
    \caption{Obtención del Léxico}
\end{algorithm}
\end{columns}
\end{frame}
%--------------Algoritmos implementados-------
\subsection{Algoritmos implementados}
\begin{frame}
\frametitle{Tokenizador}
\end{frame}

%--------------Resultados----------------
\subsection{Resultados}







\begin{frame}
\frametitle{Table}
\begin{table}
\begin{tabular}{l l l}
\toprule
\textbf{Treatments} & \textbf{Response 1} & \textbf{Response 2}\\
\midrule
Treatment 1 & 0.0003262 & 0.562 \\
Treatment 2 & 0.0015681 & 0.910 \\
Treatment 3 & 0.0009271 & 0.296 \\
\bottomrule
\end{tabular}
\caption{Table caption}
\end{table}
\end{frame}

%------------------------------------------------

%\begin{frame}
%\frametitle{Theorem}
%\begin{theorem}[Mass--energy equivalence]
%$E = mc^2$
%\end{theorem}
%\end{frame}

%------------------------------------------------

\begin{frame}[fragile] % Need to use the fragile option when verbatim is used in the slide
\frametitle{Verbatim}
\begin{example}[Theorem Slide Code]
\begin{verbatim}
\begin{frame}
\frametitle{Theorem}
\begin{theorem}[Mass--energy equivalence]
$E = mc^2$
\end{theorem}
\end{frame}\end{verbatim}
\end{example}
\end{frame}

%------------------------------------------------

\begin{frame}
\frametitle{Figure}
Uncomment the code on this slide to include your own image from the same directory as the template .TeX file.
%\begin{figure}
%\includegraphics[width=0.8\linewidth]{test}
%\end{figure}
\end{frame}

%------------------------------------------------

\begin{frame}[fragile] % Need to use the fragile option when verbatim is used in the slide
\frametitle{Citation}
An example of the \verb|\cite| command to cite within the presentation:\\~

This statement requires citation \cite{p1}.
\end{frame}

%------------------------------------------------

\begin{frame}
\frametitle{References}
\footnotesize{
\begin{thebibliography}{99} % Beamer does not support BibTeX so references must be inserted manually as below
\bibitem[Smith, 2012]{p1} John Smith (2012)
\newblock Title of the publication
\newblock \emph{Journal Name} 12(3), 45 -- 678.
\end{thebibliography}
}
\end{frame}

%------------------------------------------------

\begin{frame}
\Huge{\centerline{The End}}
\end{frame}

%----------------------------------------------------------------------------------------

\end{document} 