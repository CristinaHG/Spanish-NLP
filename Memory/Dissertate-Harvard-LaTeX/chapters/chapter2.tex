%!TEX root = ../dissertation.tex
\chapter*{Resolución del trabajo}
\begin{savequote}[75mm]
"If I were to pick a language to use today other than Java, it would be Scala."
\qauthor{James Gosling}
\end{savequote}

\chapter{Introducción a Scala}
\newthought{SCALA}, (Scalable language) es un lenguaje de programación de propósito general cuya primera versión fue lanzada en 2004, y desde entonces ha ido creciendo enormemente en usuarios. Estas son algunas razones por las que usar Scala \citet{9781491949856}:
\section*{Es escalable} Tal y como su nombre indica, este lenguaje ha sido diseñado para crecer con las demandas de sus usuarios, por lo que es una buena opción para escribir desde scripts pequeños a grandes sistemas, abordar desafíos actuales como el Big data o proporcionar servicios con gran disponibilidad y robustez. Esta es la principal razón por la que se escoge para este proyecto, dado que la intención es ir ampliándolo.

\section*{Soporta un paradigma mixto} Por una parte \textsf{Scala} soporta programación orientada a objetos (POO), mejorando los objects de Java incluyendo los \textsf{traits}, una forma clara de implementar los tipos usando composiciones mixtas. En \textsf{Scala} todo son objetos realmente, incluso los tipos numéricos. Por otro lado, también soporta totalmente programación funcional (FP),
herramienta que se ha convertido en la mejor forma de pensar en concurrencia, Big data y corrección del código en general (empleo de inmutabilidad, funciones de primera clase, funciones de alto orden...). 

\section*{Tiene un sofisticado sistema de tipos} Extiende el sistema de tipos de Java con otros tipos genéricos más flexibles y otras mejoras para mejorar la corrección del código. Además \textsf{Scala} incorpora un mecanismo de inferencia de tipos.
\section*{Es estaticamente tipado} \textsf{Scala} incorpora el tipado estático como herramienta para crear aplicaciones más robustas, pero añade algunas modificaciones para hacerlo más llevadero, como incorporar la inferencia de tipos y hacerlo más flexible, permitiendo identificación de patrones y nuevas formas de escribir y componer tipos.
\section*{Un lenguaje JVM y Javascript} Explota las funcionalidades y optimizaciones de JVM, así como la gran cantidad de liberías y herramientas disponibles para Java. Además tiene un puerto para JavaScript (Scala.js).
\section*{Sintaxis concisa, elegante y flexible} Si de algo hablan los programadores de Scala es de su sintaxis y de las reducciones de código que experimentan con respecto a Java.  Por ejemplo:
\begin{minted}
[frame=lines,
framesep=1mm,
fontsize=\footnotesize
]{java}
 // código en Java
  class Persona {
      private int edad;
      private String nombre;
  
      public Persona(int edad, String nombre) {
          this.edad = edad;
          this.nombre = nombre;
      }
  }
  
  // código en Scala
    class Persona(edad: Int, nombre: String)
\end{minted}
Sí, dado este código en Scala el compilador creará dos atributos privados edad(entero) y nombre(cadena de caracteres) y un constructor que tomará los valores que se le pasen inicialmente para inicializar esas variables. Más rápido de escribir, de leer, y menos errores.  
\section*{Más ejemplos de Scala}
A continuación se muestran algunos ejemplos simples de \textsf{Scala} obtenidos de \citet{9781935182757}
\subsection{Encontrar caracter en mayúscula}
Éste es un ejemplo en Java y Scala del problema de encontrar un caracter en mayúcula en una cadena de texto:
\begin{minted}
[frame=lines,
framesep=1mm,
fontsize=\footnotesize
]{java}
 // código en Java
  boolean hasUpperCase = false;
for(int i = 0; i < name.length(); i++) {
    if(Character.isUpperCase(name.charAt(i))) {
        hasUpperCase = true;
        break;  
    }
}
 // código en Scala
    val hasUpperCase = name.exists(_.isUpper)
\end{minted}
Mientras que el código en Java itera sobre cada caracter del string, comprobando uno a uno hasta que encuentra una mayúscula, modifica el flag boolenano y se sale del bucle, en Scala basta con llamar a la función \textsf{exist} sobre el string \textsf{name}, devolviendo un booleano. La \textsf{\_} representa cada caracter de la cadena. De nuevo Scala seduce con su sintaxis. 
\subsection{Contar las líneas de un fichero}
\begin{minted}
[frame=lines,
framesep=1mm,
fontsize=\footnotesize
]{ruby}
 # código en Ruby
count = 0     
File.open "someFile.txt" do |file|
   file.each { |line| count += 1 }
end 
\end{minted}
\begin{minted}
[frame=lines,
framesep=1mm,
fontsize=\footnotesize
]{java}
 // código en Scala
val count = scala.io.Source.fromFile("someFile.txt").getLines().map(x => 1).sum 
\end{minted}
En \textsf{Ruby} también se realiza de forma breve, pero no tan elegante, ya que necesita una variable contador que va incrementando en cada vez que cuenta una línea. En \textsf{Scala}, una posible forma de hacerlo es usando \textsf{map}, una función que a cada línea del fichero le hace corresponder un 1. Finalmente los suma todos llamando a \textsf{sum}. \newline
Además, \textsf{Scala} soporta composición mixta a través de los \textsf{traits}. Los \textsf{traits} son parecidos a las clases abstractas con implementación parcial. Por ejemplo, podríamos crear un nuevo tipo de colección que permitiera acceder al contenido del fichero como un \textsf{iterable}, mezclando el trait \textsf{Iterable} de \textsf{Scala}:
\begin{minted}
[frame=lines,
framesep=1mm,
fontsize=\footnotesize
]{java}
 class FileAsIterable {
  def iterator = scala.io.Source.fromFile("someFile.txt").getLines()
}
\end{minted}
Ahora si lo mezclamos con el trait \textsf{Iterable} de \textsf{Scala}, al crear un objeto de esa clase, éste será un \textsf{Iterable}, teniendo acceso a los métodos de \textsf{Iterable}:
 \begin{minted}
[frame=lines,
framesep=1mm,
fontsize=\footnotesize
]{java}
val newIterator = new FileAsIterable with Iterable[String]
newIterator.foreach { line => println(line) }
\end{minted}
donde el método \textsf{foreach} al que llama, es de \textsf{Iterable}.
\section*{Scala y la concurrencia}
Uno de los grandes problemas de la concurrencia es que si no se coordina bien el acceso a los recursos puede haber cambios inesperados e indeseados en los mismos, por acción de alguna de las hebras. \textsf{Scala} ayuda en esto con la inmutabilidad, de hecho, por defecto lo hace todo inmutable. Aunque en Scala se puede usar cualquier mecanismo de Java, Scala incorpora también sus propias herramientas específicas para concurrencia. Algunas de ellas son: \citet{9781491949856} 
\subsection{Futures}
La API \textcolor{SchoolColor}{scala.concurrent.Future} simplifica la concurrencia en código. Los Futures empiezan a correr concurrentemente cuando son creados, aunque lo hacen de forma asíncrona. Se puede hacer que las tareas sean independientes y sin bloqueo o por el contrario, bloquear, y además la API ofrece muchas funcionalidades para manejar los resultados (que pueden ser un Future), como callbacks que pueden ser invocados cuando el resultado esté listo. Veamos un ejemplo simple en el que se lanzan concurrentemente 5 tareas:  
\begin{minted}
[frame=lines,
framesep=1mm,
fontsize=\footnotesize
]{java}
import scala.concurrent.Future
import scala.concurrent.ExecutionContext.Implicits.global
def sleep(millis: Long) = {
Thread.sleep(millis)
}
// Busy work ;)
def doWork(index: Int) = {
sleep((math.random * 1000).toLong)
index
}
(1 to 5) foreach { index =>
val future = Future {
doWork(index)
}
  future onSuccess {
    case answer: Int => println(s"Success! returned: $answer")  
  }
  future onFailure {
    case th: Throwable => println(s"FAILURE! returned: $th")
  }
}
sleep(1000) // Wait long enough for the "work" to finish.
println("Finito!")
\end{minted}
Se usa un método \textsc{sleep} para simular la ocupación por un tiempo. El método \textsc{doWork} llama a sleep con un número aleatorio de milisegundos. Se itera con \textsf{foreach} en un rango de enteros (1 a 5 inclusive) y se llama a \textcolor{SchoolColor}{scala.concurrent.Future.apply}. En este caso, \textcolor{SchoolColor}{Future.apply} recibe una función de trabajo a realizar (doWork), y devuelve un nuevo objeto\textcolor{SchoolColor}{Future}, el cual ejecuta el cuerpo de doWork(index) en otra hebra, devolviendo el control inmediatamente al bucle. Finalmente se usan \textcolor{SchoolColor}{onSucess} (si la tarea se completa correctamente) y \textcolor{SchoolColor}{onFailure} (si falla) para registrar los callbacks. 
\subsection{Modelo de actores}
Una regla de la concurrencia es: si puedes, no compartas. Los actores son entidades software independientes que no comparten unos con otros información mutable que se pueda alterar. En su lugar se comunican mediante mensajes, eliminando la necesidad de sincronizar el acceso a datos, estados e información mutable. De esta forma es más fácil crear aplicaciones concurrentes robustas. Cada actor cambia su estado según lo necesite, pero sólo si tiene acceso exclusivo a ese estado y sus invocaciones se garantizan seguras. 

\section*{Scala y Big data}
Las funciones \textsf{map, flatMap, filter, fold, reduce...etc} siempre han sido funciones para trabajar con datos, independientemente de que estos fueran grandes o pequeños. Es por esto que una vez que se entiende \textsf{Scala} y sus colecciones, resulta fácil enganchar una API de Big data basada en Scala, sin embargo, sabiendo Java, enganchar con MapReduce Java API resulta más complicado por ser más difícil de usar y estar a más bajo nivel.\newline
Pero la principal ventaja de Scala en Big data frente a otros lenguajes no es la curva de aprendizaje, sino la programación funcional. Scala permite escribir lo mismo (o más), con menos código que otros lenguajes. El ejemplo simple de contar el número de palabras, sacado de \citet{9781491949856} bastará para evidenciar a simple vista la ventaja de las APIs Big data basadas en Scala frente a las basadas en Java. \newline
Esto es sólo parte de la versión \textsf{Hadoop}. El original se compone de unas 60 líneas de código, sin imports ni comentarios.
\begin{minted}
[frame=lines,
framesep=1mm,
fontsize=\footnotesize
]{java}
// src/main/java/progscala2/bigdata/HadoopWordCount.javaX
...
class WordCountMapper extends MapReduceBase
   implements Mapper<IntWritable, Text, Text, IntWritable> {
   
 static final IntWritable one = new IntWritable(1);
 static final Text word = new Text;
 
 @Override public void map(IntWritable key, Text valueDocContents,
   OutputCollector<Text, IntWritable> output, Reporter reporter) {
  String[] tokens = valueDocContents.toString.split("\\s+");
  for (String wordString: tokens) {
    if (wordString.length > 0) {
      word.set(wordString.toLowerCase);
      output.collect(word, one);
    }
  }
 }
}

class WordCountReduce extends MapReduceBase
  implements Reducer<Text, IntWritable, Text, IntWritable> {
  
public void reduce(Text keyWord, java.util.Iterator<IntWritable> counts,
   OutputCollector<Text, IntWritable> output, Reporter reporter) {
 int totalCount = 0;
 while (counts.hasNext) {
   totalCount += counts.next.get;
}
\end{minted}

Ahora una versión \textsf{Scalding}, basado en Scala. El código orignial tiene 12 líneas de código, contando el import:
\begin{minted}
[frame=lines,
framesep=1mm,
fontsize=\footnotesize
]{scala}
// src/main/scala/progscala2/bigdata/WordCountScalding.scalaX
import com.twitter.scalding._

class WordCount(args : Args) extends Job(args) {

  TextLine(args("input"))
     .read
     .flatMap('line -> 'word) {
       line: String => line.trim.toLowerCase.split("""\s+""")
     }
     .groupBy('word){ group => group.size('count) }
     .write(Tsv(args("output")))
}
\end{minted}
