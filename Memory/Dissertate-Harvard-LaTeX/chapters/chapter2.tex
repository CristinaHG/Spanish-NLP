%!TEX root = ../dissertation.tex
\begin{savequote}[75mm]
This is some random quote to start off the chapter.
\qauthor{Firstname lastname}
\end{savequote}

\chapter{Resolución del trabajo}
\chapter*{Introducción a Scala}
\newthought{SCALA (Scalable language)}, es un lenguaje de programación de propósito general cuya primera versión fue lanzada en 2004, y desde entonces ha ido creciendo enormemente en usuarios. Estas son algunas razones por las que usar Scala:
\section*{Es escalable} Tal y como indica su nombre, este lenguaje ha sido diseñado para crecer con las demandas de sus usuarios, por lo que es una buena opción para escribir desde scripts pequeños a grandes sistemas, abordar desafíos actuales como el Big data o proporcionar servicios con gran disponibilidad y robustez.

\section*{Soporta un paradigma mixto} Por una parte \textsf{Scala} soporta programación orientada a objetos (POO), mejorando los objects de Java incluyendo los \textsf{traits}, una forma clara de implementar los tipos usando composiciones mixtas. En \textsf{Scala} todo son objetos realmente, incluso los tipos numéricos. Por otro lado, también soporta totalmente programación funcional (FP),
herramienta que se ha convertido en la mejor forma de pensar en concurrencia, Big data y corrección del código en general (empleo de inmutabilidad, funciones de primera clase, funciones de alto orden...). 

\section*{Tiene un sofisticado sistema de tipos} Extiende el sistema de tipos de Java con otros tipos genéricos más flexibles y otras mejoras para mejorar la corrección del código. Además \textsf{Scala} incorpora un mecanismo de inferencia de tipos.
\section*{Es estaticamente tipado} \textsf{Scala} incorpora el tipado estático como herramienta para crear aplicaciones más robustas, pero añade algunas modificaciones para hacerlo más llevadero, como incorporar la inferencia de tipos y hacerlo más flexible, permitiendo identificación de patrones y nuevas formas de escribir y componer tipos.
\section*{Un lenguaje JVM y Javascript} Explota las funcionalidades y optimizaciones de JVM, así como la gran cantidad de liberías y herramientas disponibles para Java. Además tiene un puerto para JavaScript (Scala.js).
\section*{Sintaxis concisa, elegante y flexible} Si de algo hablan los programadores de Scala es de su sintaxis y de las reducciones de código que experimentan con respecto a Java.  Por ejemplo:
\begin{minted}
[frame=lines,
framesep=1mm,
fontsize=\footnotesize
]{java}
 // código en Java
  class MyClass {
      private int index;
      private String name;
  
      public MyClass(int index, String name) {
          this.index = index;
          this.name = name;
      }
  }
\end{minted}


