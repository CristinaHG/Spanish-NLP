%!TEX root = ../dissertation.tex
\begin{savequote}[75mm]
Nulla facilisi. In vel sem. Morbi id urna in diam dignissim feugiat. Proin molestie tortor eu velit. Aliquam erat volutpat. Nullam ultrices, diam tempus vulputate egestas, eros pede varius leo.
\qauthor{Quoteauthor Lastname}
\end{savequote}

\chapter{Planificación, análisis y diseño}
\chapter*{Planificación}
A continuación se expone mediante un \textcolor{SchoolColor}{diagrama de Gantt} como se ha distribuido el tiempo durante estos 5 meses para cada tarea de este proyecto, desde mediados de Julio a la segunda semana de diciembre.
\\[\baselineskip]
%\\[\baselineskip]
\setganttlinklabel{f-s}{FINISH-TO-START}
\begin{figure}[H]
 \makebox[\textwidth][c]{
\begin{ganttchart}[ vgrid,x unit=0.4cm,
            y unit title=0.7cm,
            y unit chart=0.9cm,
            canvas/.append style={rounded corners=1mm, draw=none, fill=orange!9},
            bar/.append style={fill=orange!30,rounded corners=1.5mm, draw=none},
            milestone/.append style={fill=orange!90,rounded corners=0mm, draw=none},
            link/.append style={-to,black!70},
            title/.style={fill=orange!60, draw=none, rounded corners=2mm},
title label font=\color{white}\bfseries,
title left shift=-.1,
title right shift=-.1,
title top shift=.05,
title height=.75
]{1}{30}
\gantttitle{2016}{30} \\
%\gantttitlelist{1,...,6}{5} \\
\gantttitle{Julio}{5}
\gantttitle{Agosto}{5}
\gantttitle{Septiembre}{5}
\gantttitle{Octubre}{5}
\gantttitle{Noviembre}{5}
\gantttitle{Diciembre}{5} \\
%\ganttgroup{Group 1}{1}{7} \\
\ganttbar{Revisión bibliográfica y documentación}{3}{9} \\
\ganttlinkedbar[link type=f-s]{Análisis de requisitos y diseño}{10}{10}\\
\ganttlinkedbar[link type=f-s]{Implementación y eval. Del Tokenizador}{11}{15}\\
\ganttlinkedbar[link type=f-s]{Implementación y eval. Del POS Tagger}{16}{20}\\
\ganttlinkedbar[link type=f-s]{Implementación y eval. Del Lematizador}{21}{25}\\
\ganttmilestone{Fin de código}{25} \ganttnewline
\ganttlinkedbar{Memoria}{26}{27}\\
%\ganttlinkedbar[link type=f-s]{Interfaz gráfica}{19}{23}\\
%\ganttmilestone{Informe trimestral 2}{24} \ganttnewline
%\ganttlinkedbar{Interfaz gráfica}{23}{29}\\
%\ganttlinkedbar{Entrega final}{29}{30}\\
%\ganttlink{elem3}{elem4}
\ganttlink{elem4}{elem5}
\end{ganttchart}}
\end{figure} 

\chapter*{Análisis de requisitos}
En esta sección se muestra la especificación de requisitos para el proyecto de manera formal, a través de plantillas que muestran información relevante del requisito, como la descripción, su prioridad, estado o riesgo. Al ser demasiados, se exponen solo los más importantes.

\begin{table}[H]
\centering
%\caption{My caption}
\label{my-label}
\begin{tabular}{|l|l|l|l|l|l|l|}
\hline
\multicolumn{7}{|l|}{Identificador:}                                 \\ \hline
\multicolumn{4}{|l|}{Necesidad:} & \multicolumn{3}{l|}{Autor:}         \\ \hline
\multicolumn{7}{|l|}{Descripción:}                                 \\ \hline
\multicolumn{4}{|l|}{Prioridad:} & \multicolumn{3}{l|}{Riesgo:}         \\ \hline
\multicolumn{5}{|l|}{Dependencias:}         & \multicolumn{2}{l|}{Estado:} \\ \hline
\end{tabular}
\end{table}










