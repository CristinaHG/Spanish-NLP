%!TEX root = ../dissertation.tex
%\begin{savequote}[75mm]
%Nulla facilisi. In vel sem. Morbi id urna in diam dignissim feugiat. Proin molestie tortor eu velit. %Aliquam erat volutpat. Nullam ultrices, diam tempus vulputate egestas, eros pede varius leo.
%\qauthor{Quoteauthor Lastname}
%\end{savequote}

\chapter{Objetivos del trabajo}
En este trabajo se implementan las tres primeras herramientas para un software de PLN en Español. El trabajo se implementa usando el lenguaje \textcolor{SchoolColor}{Scala}, a partir de las propuestas de \citet{smedt2012pattern} y se puede consultar una planificación temporal de los objetivos del trabajo en la parte de \textsf{resolución del trabajo}, en el capítulo 5, sección de Planificación. Igualmente se puede consultar una introducción a \textcolor{SchoolColor}{Scala} en esta parte, en el capítulo 4.\newline Los objetivos del trabajo se detallan brevemente a continuación: 
\subsection*{Documentación y revisión bibliográfica}
Aquí el objetivo es adquirir conocimiento general sobre la temática del procesamiento del lenguaje natural, tokenización, lematización y pos etiquetado para Español, documentarse sobre qué trabajos se han realizado anteriormente, la metodología a seguir, qué trabajos hay ahora y el estado del arte. 
\subsection*{Elección de técnicas a implementar y lenguaje de programación, planificación del trabajo, análisis de requisitos y diseño}
El segundo objetivo consiste en:
\begin{itemize}
\item elegir tres técnicas: de tokenización, Pos tagger y lematización respectivamente. Finalmente se decide implementar las tres técnicas para estas herramientas que presenta \citet{smedt2012pattern} para español.
\item elegir el lenguaje de programación: se elige el lenguaje Scala para la implementación de este trabajo por sus múltiples ventajas a pesar de que su estructura resulta inicialmente compleja. Esta decisión se detalla en el capítulo resolución del trabajo, en la parte de Introducción a Scala.
\item realizar un análisis de los requisitos y de diseño de las herramientas a implementar y su integración conjunta. Se detalla dentro del capítulo de resolución del trabajo, en la parte de Análisis y diseño.
\end{itemize}

\subsection*{Implementación en código, integración y evaluación de las técnicas implementadas}
Aquí se implementan los diferentes módulos de tokenización, etiquetado morfosintáctico y parseo, siguiendo un desarrollo guiado por pruebas (TDD) escribiendo primero los test, usando para ello la librería Scalatest y el estilo FunSuite. Esto se detalla en el capítulo de resolución del trabajo, en la parte de implementación y pruebas. 

\subsection*{Análisis y exposición de resultados, comparaciones y conclusión}   
Aquí se discuten y analizan los resultados obtenidos, comparandolos con los del trabajo original y con los de otros softwares de PNL disponibles, (capítulo de marco experimental y resultados ) y se exponen vías futuras del mismo (capítulo conclusiones y vías futuras).


