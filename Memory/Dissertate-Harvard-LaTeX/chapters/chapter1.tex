%!TEX root = ../dissertation.tex
\begin{savequote}[75mm]
Nulla facilisi. In vel sem. Morbi id urna in diam dignissim feugiat. Proin molestie tortor eu velit. Aliquam erat volutpat. Nullam ultrices, diam tempus vulputate egestas, eros pede varius leo.
\qauthor{Quoteauthor Lastname}
\end{savequote}

\chapter{Objetivos del trabajo}
Los objetivos del trabajo se han dividido en: 
\section{Documentación y revisión bibliográfica}
Aquí el objetivo es adquirir conocimiento general sobre la temática del procesamiento del lenguaje natural, tokenización, lematización y Pos etiquetado para Español, documentarse sobre qué trabajos se han realizado anteriormente, la metodología a seguir, qué trabajos hay ahora y el estado del arte. 
\section{Elección de técnicas a implementar y lenguaje, análisis de requisitos y diseño}
El segundo objetivo consiste en:
\begin{itemize}
\item elegir tres técnicas: de tokenización, Pos tagger y lematización respectivamente. Finalmente se decide implementar las tres técnicas para estas herramientas que presenta \citet{smedt2012pattern} para español.
\item elegir el lenguaje de programación: se elige el lenguaje Scala para la implementación de este trabajo por sus múltiples ventajas a pesar de que su extructura resulta inicialmente compleja. Esta decisión se detalla en el siguiente capítulo.
\item realizar un análisis de los requisitos y de diseño de las herramientas a implementar y su integración conjunta.
\end{itemize}

\section{Implementación en código, integración y evaluación de las técnicas implementadas}
Aquí se implementan los diferentes módulos de tokenización, etiquetado morfosintáctico y parseo, siguiendo un desarrollo guiado por pruebas (TDD) escribiendo primero los test, usando para ello la librería Scalatest y el estilo FunSuite. 

\section{Análisis y exposición de resultados, comparaciones y conclusión}   
Aquí se discuten y analizan los resultados obtenidos, comparandolos con los del paper original y con los de otros softwares de NLP disponibles, y se exponen vías futuras del mismo. 


