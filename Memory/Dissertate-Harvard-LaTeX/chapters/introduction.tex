%!TEX root = ../dissertation.tex
\chapter{Introducción y motivación}
\label{introducción}
El proyecto desarrollado se engloba en el campo del Procesamiento del lenguaje natural. El lenguaje natural es cualquier lenguaje usado por los humanos para comunicarse (Alemán, Inglés, Español, Hindi...). Dado que estos lenguajes se transmiten entre generaciones, van experimentando evoluciones, lo que hace difícil obtener reglas que los describan. (PLN) es, por tanto, el área de estudio y aplicación que engloba cualquier tipo de manipulación computacional del lenguaje natural.\newline Es decir, PNL abarca desde aplicaciones simples, como contar el número de ocurrencias de las palabras en un texto para comparar diferentes estilos de escritura, a aplicaciones más complejas, como comprender expresiones humanas completas para poder dar respuestas útiles a preguntas. \citet{bird2009natural} Como por ejemplo, el asistente Siri de iPhone. 

La linguística computacional o PLN comenzó en 1980, sin embargo en los últimos 20 años ha crecido enormemente, despertando un gran interés en el ámbito de la investigación científica pero también en el ámbito práctico, ya que cada vez son más los productos, especialmente los tecnológicos, que incorporan algún tipo de aplicación basada en NLP. Por ejemplo, traductores como el traductor de Skype, o asistentes de voz inteligentes (Cortana de Microsoft, Google Now de Google o el ya mencionado Siri de Apple). \newline
Este crecimiento en el campo del procesamiento del lenguaje natural se debe principalmente a que en los últimos años, con el uso de redes sociales como Facebook, SnapChat, Twitter, Google plus, Linked in... y de sitios web comerciales como Amazon o Booking, los usuarios han generado una gran cantidad de contenido mayoritariamente subjetivo, el cual se puede aplicar en muchos ámbitos como márketing, política, gestión de crisis, soporte, atención al cliente, etc. También han influido en su crecimiento el aumento de capacidad de procesamiento y cómputo que ha habido en los últimos años y el desarrollo de técnicas de machine learning más complejas y potentes.    
\newline
Actualmente, según lo descrito en \citet{hirschberg2015advances} éstas son algunas de las principales áreas en PLN:
\section*{Traducción automática}
La traducción automática es el área del PLN que estudia el empleo de sofware para ayudar a traducir de un lenguaje natural a otro, ya sea en texto o hablado. Ésto supone una gran dificultad, ya que para que una traducción sea correcta, no basta con traducir palabra a palabra, sino que hay que tener en cuenta el sentido de la palabra y el contexto de ésta, pues hay casos en los que la misma palabra significa varias cosas dependiendo del contexto. Por ejemplo, en las frases \textsl{"compra una lata de refresco"} y \textsl{"deja ya de dar la lata"} aparece la  palabra \textsl{lata} desempeñando una  función distinta:\newline En la primera frase, \textsl{lata} es un nombre, por lo que se entendería como un envase hecho de hojalata, mientras que en la segunda frase aparece como una locución verbal, por lo que se entendería como "molestar" o "importunar". \newline  
El campo de la traducción automática se empezó a estudiar a finales de 1950s, sin embargo inicialmente no tuvo mucho éxito debido a que los traductores construidos eran sistemas basados en gramáticas escritas a mano. Fue a partir de 1990 y gracias a que los científicos de IBM consiguieron una cantidad suficientemente grande de frases de traducciones entre dos lenguajes, cuando construyeron un modelo probabilístico de traducción automática.\newline
A partir de entonces se siguió investigando y se descubrieron los \textcolor{SchoolColor}{traductores máquina basados en frases}, que en lugar de ir traduciendo palabra a palabra, detectaban los pequeños subgrupos de palabras que solían ir juntas y que tenían una traducción especial. Esto se utilizó para desarrollar el traductor de Google.  

Actualmente, el estado del arte en este campo está en traductores máquinas que usan deep learning, entrenando un modelo de varios niveles para optimizar un objetivo (la calidad de la traducción), donde luego el modelo pueda aprender por sí mismo más niveles que le sean útiles para desarrollar la tarea. Esto ha sido estudiado especialmente para redes neuronales, habiendo conseguido en varios casos obtener los mejores resultados hasta el momento, empleando redes neuronales distribuidas. Como por ejemplo, en  \citet{luong2014addressing}.

Cras dictum. Maecenas ut turpis. In vitae erat ac orci dignissim eleifend. Nunc quis justo. Sed vel ipsum in purus tincidunt pharetra. Sed pulvinar, felis id consectetuer malesuada, enim nisl mattis elit, a facilisis tortor nibh quis leo. Sed augue lacus, pretium vitae, molestie eget, rhoncus quis, elit. Donec in augue. Fusce orci wisi, ornare id, mollis vel, lacinia vel, massa.

Lorem ipsum dolor sit amet, consectetuer adipiscing elit. Morbi commodo, ipsum sed pharetra gravida, orci magna rhoncus neque, id pulvinar odio lorem non turpis. Nullam sit amet enim. Suspendisse id velit vitae ligula volutpat condimentum. Aliquam erat volutpat. Sed quis velit. Nulla facilisi. Nulla libero. Vivamus pharetra posuere sapien. Nam consectetuer. Sed aliquam, nunc eget euismod ullamcorper, lectus nunc ullamcorper orci, fermentum bibendum enim nibh eget ipsum. Donec porttitor ligula eu dolor. Maecenas vitae nulla consequat libero cursus venenatis. Nam magna enim, accumsan eu, blandit sed, blandit a, eros.

Quisque facilisis erat a dui. Nam malesuada ornare dolor. Cras gravida, diam sit amet rhoncus ornare, erat elit consectetuer erat, id egestas pede nibh eget odio. Proin tincidunt, velit vel porta elementum, magna diam molestie sapien, non aliquet massa pede eu diam. Aliquam iaculis. Fusce et ipsum et nulla tristique facilisis. Donec eget sem sit amet ligula viverra gravida. Etiam vehicula urna vel turpis. Suspendisse sagittis ante a urna. Morbi a est quis orci consequat rutrum. Nullam egestas feugiat felis. Integer adipiscing semper ligula. Nunc molestie, nisl sit amet cursus convallis, sapien lectus pretium metus, vitae pretium enim wisi id lectus. Donec vestibulum. Etiam vel nibh. Nulla facilisi. Mauris pharetra. Donec augue. Fusce ultrices, neque id dignissim ultrices, tellus mauris dictum elit, vel lacinia enim metus eu nunc.
