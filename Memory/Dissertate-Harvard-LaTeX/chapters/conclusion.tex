%!TEX root = ../dissertation.tex
\chapter{Conclusiones y vías futuras}
\label{conclusion}
\section{Conclusiones}
En este trabajo se implementan e integran tres algoritmos en el lenguaje \textsc{Scala} que resuelven las tres primeras fases de una herramienta para procesamiento del lenguaje natural. En concreto, se desarrolla un algoritmo de tokenización, otro de Pos etiquetado y otro de lematización, probando que no hace falta utilizar el algoritmo más complejo existente para ser capaz de dar unos resultados decentes.\newline En todos los algoritmos, especialmente en el lematizador, se hace uso de muchas reglas conocidas existentes en nuestro lenguaje para implementar el algoritmo. Aunque es imposible cubrir todos los posibles casos con reglas, se pueden obtener resultados atractivos sólo empleando algunas de ellas. 
\section{Vías futuras}
Como vías futuras se podría implementar la extracción de los ficheros de léxico, morfología y contexto que aplica el Pos tagger, en lugar de tomarlos ya construidos. Concretamente para el contexto se podría implementar un algoritmo de Brill en Scala que se incluya en la herramienta, y algún otro algoritmo más potente que hiciera posible una comparativa. 
\\[\baselineskip]
Otras posibles mejoras son añadir manejo de emoticonos en el tokenizador, intentar hacer todo el código inmutable, dado que \textsc{Scala} anima a ello pero es tan complicado visualizarlo al principio que, de haberlo hecho en este trabajo, no nos habría dado tiempo a cumplir los plazos, y seguir ampliando el trabajo con el resto de módulos de PLN hasta tener un software completo para análisis de sentimientos para español.