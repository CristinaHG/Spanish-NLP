%!TEX root = ../dissertation.tex
% the abstract

In this work is exposed a \textsc{Scala} implementation of three initial and 	essential algorithms in natural language processing tasks. On the first hand, a tokenization algorithm, that iterates whithin three steps extracting tokens whithin a given text and its sentences is exposed. On the second hand, is exposed a POS Tagger algorithm that tags all the extracted words in the previus step by their morphosyntactic information, using \textsf{Parole} tagset by default. In order to do this, tagger applies lexical, morphological and contextual rules, where contextual ones are acquired by a Brill algorithm. In last place, a lemmatizer that gets the basic form of given words is exposed. In order to achieve that, it takes conscience about given word's tags in previous tagging proccess to apply one lemmatization rules set or another.\newline
The three algorithm are developed for Spanish language, achieving similar results and sometimes quite superior results to the original implementation proposed in Spanish version of \citet{smedt2012pattern} (\textsc{Pattern.es}).
Last but not at least, the three algorithms are integrated whithin a parser, so it could be possible to taste one or all the implemented algorithms by command line.   
\newline

